\documentclass[12pt,a4paper]{article}
\usepackage[latin1]{inputenc}
\usepackage{amsmath}
\usepackage{amsfonts}
\usepackage{amssymb}
\usepackage{graphicx}
\begin{document}
	As proposed by G�tte et al. (2015), a mixture distribution which consists of the weighted sum of two normal distributions may be used as prior for the true treatment effect for the time-to-event endpoint $\theta = -\log(HR)$ ($HR$, hazard ratio):
	\[ \theta \sim w \cdot \mathcal{N}(-\log(HR_1),(4/i^d_1)) + (1-w) \cdot \mathcal{N}(-\log(HR_2),(4/i^d_2)), \]
	where $\mathcal{N}(\mu,\sigma^2)$ denotes the normal distribution with mean $\mu$ and variance $\sigma^2$. The two normal distributions each depict a distribution for $\theta$ with the first one representing e.g. a strong and the second one e.g. a moderate treatment effect, whereby the denominators of the variances can be viewed as \textquotedblleft amount of information\textquotedblright{} about the treatment effect size in terms of numbers of events.\\
	
	To model the true treatment effect measured by the risk ratio $RR = p_1/p_0$, the control rate $p_0$ is assumed fixed and the rate of the treatment group $p_1$ may be modelled by the weighted sum of two normal distributions approximating binomial distributions
	\[p_1 \sim w \cdot \mathcal{N}\bigg(p_{1,1},\frac{p_{1,1}\cdot(1-p_{1,1})}{i^n_1}\bigg) + (1-w) \cdot \mathcal{N}\bigg(p_{1,2},\frac{p_{1,2}\cdot(1-p_{1,2})}{i^n_2}\bigg), \]
	where $\mathcal{N}(\mu,\sigma^2)$ denotes the normal distribution with mean $\mu$ and variance $\sigma^2$. The two normal distributions each depict a distribution for $p_1$ with the first one representing e.g. a lower and the second one e.g. a higher rate in the treatment group, whereby the denominators of the variances can be viewed as \textquotedblleft amount of information\textquotedblright{} about the rate in terms of sample size.\\
	
	A prior distribution for the standardized true difference in means ($\Delta$) may be given by the weighted sum of two truncated normal distributions
	\[ \Delta \sim w \cdot \mathcal{N}^{t}_{[a, b]}(\Delta_2, 4/i^n_1) + (1-w) \cdot \mathcal{N}^{t}_{[a, b]}(\Delta_1, 4/i^n_2), \]
	where $\mathcal{N}^t_{[a,b]}(\mu,\sigma^2)$ denotes the truncated normal distribution with mean $\mu$ and variance $\sigma^2$, truncated below $a$ and above $b$ (compare Rufibach et al., 2016). The two truncated normal distributions each depict a distribution for $\Delta$ with the first one representing e.g. a strong and the second one e.g. a moderate treatment effect, whereby the denominators of the variances can be viewed as \textquotedblleft amount of information\textquotedblright{} about the treatment effect size in terms of sample size.
	


	To model different population structures in phase II and III assume different distributions for the assumed true treatment effect in phase II ($\theta_2$) and III ($\theta_3$), e.g. $\theta_3\sim \theta_2+\gamma$.
	
	
	To model different population structures in phase II and III assume different distributions for the assumed true treatment effect in phase II ($\Delta_2$) and III ($\Delta_3$), e.g. $\Delta_3\sim \Delta_2+\gamma$.
	
	
	
	To model different population structures in phase II and III assume different distributions for the assumed true treatment effect in phase II ($p_{1, 2}$) and III ($p_{1, 3}$), e.g. $p_{1, 3}\sim p_{1, 2}-\gamma$.
	

\end{document}